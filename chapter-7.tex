\documentclass[a4paper,11pt]{article}
\usepackage[utf8]{inputenc}
\usepackage{amssymb}
\usepackage{amsmath} 
\usepackage{enumerate}

\DeclareMathOperator*{\argmax}{arg\!\max}
\DeclareMathOperator*{\argmin}{arg\!\min}
\newcounter{exercise}
\setcounter{exercise}{0}
\newcounter{subexercise}
\newcommand*{\exercise}[1][]{\subsection*{Exercise \ifx/#1/\stepcounter{exercise}\arabic{exercise}\else#1\fi}\setcounter{subexercise}{0}}
\newcommand*{\subexercise}[1][]{
\par{\noindent\ifx/#1/\protect\stepcounter{subexercise}\alph{subexercise}\else#1\fi.\quad}}

\title{Chapter 7}
\author{stevenjin8}
\date{\today}

\begin{document}
\maketitle
\section{Proofs}
\subsection*{Equation 7.54}
One of the results that I did not find trivial was equation 7.54.
Perhaps it was proven somewhere in Chapter 4.
I prove the in this subsection. As a reminder, the equation in question is
\[
    p(\mathbf{y}|\mathbf{X}, \mathbf{w}, \sigma^2) \propto
\exp(-\frac{1}{2\sigma^2}\|\mathbf{y} - \bar{y}\mathbf{1}_N - \mathbf{Xw}\|_2^2).
\]
As the author says, we must "integrate [$\mu$] out":
\[
    p(\mathbf{y}|\mathbf{X}, \mathbf{w}, \sigma^2) = 
    \int p(\mathbf{y},\mu|\mathbf{X}, \mathbf{w}, \sigma^2) d\mu.\\
\]
Since $\mu$ is independent and $p(\mu)$ is constant for all $\mu \in \mathbb{R}$,
\begin{align*}
p(\mathbf{y}|\mathbf{X}, \mathbf{w}, \sigma^2) & \propto
\int p(\mathbf{y}|\mathbf{X}, \mathbf{w}, \mu, \sigma^2)d\mu \\
& \propto \int\exp\left(\frac{
\left(\mathbf{y} - \mu\mathbf{1}_N -\mathbf{Xw}\right)^2 
}{-2\sigma^2}\right)d\mu.
\end{align*}
In this section, $\mathbf{v}^2 = \mathbf{v}.\mathbf{v}$ for any vector $\mathbf{v}$.

Next, we expand and take all terms that are independent of $\mu$ out of the
integral:
\begin{align*}
p(\mathbf{y}|\mathbf{X}&, \mathbf{w}, \sigma^2) \propto \\
\int\exp&\left(\frac{\mathbf{y}^2 - 2\mathbf{y}.(\mu\mathbf{1})
+(\mu\mathbf{1})^2 + 2(\mu\mathbf{1}).(\mathbf{Xw})
+ (\mathbf{Xw})^2 + 2(\mathbf{Xw}).\mathbf{y}}
{-2\sigma^2}\right) d\mu \\
\propto \exp&\left(\frac{\mathbf{y}^2 + 2(\mathbf{Xw}).\mathbf{y} +
(\mathbf{Xw})^2}{-2\sigma^2}\right) \\
&\int\exp\left(\frac{-2\mathbf{y}.(\mu\mathbf{1}) + (\mu\mathbf{1})^2 + 2
(\mu\mathbf{1}).(\mathbf{Xw})}
{-2\sigma^2}\right)d\mu.
\end{align*}

Since $\sum_ix_{ij}=0$, we can ignore
$\mu\mathbf{1}.\mathbf{Xw}$, and
complete the square:
\begin{equation}\begin{split}
&\int\exp\left(\frac{-2\mathbf{y}.(\mu\mathbf{1}) + (\mu\mathbf{1})^2 + (\mu\mathbf{1}).(\mathbf{Xw})}
{-2\sigma^2}\right)d\mu \\
&=\int\exp\left(\frac{N(-2\mu\bar{y} + \mu^2)}{-2\sigma^2}\right)d\mu\\
&= \exp\left(\frac{-N\bar{y}^2  }{-2\sigma^2}\right)
\int\exp\left(\frac{N(\mu - \bar{y})^2}{-2\sigma^2}\right)d\mu\\
&\propto \exp\left(\frac{-N\bar{y}^2  }{-2\sigma^2}\right) \\
&= \exp\left(\frac{-(\bar{y}\mathbf{1})^2}{-2\sigma^2}\right)
\end{split}\end{equation}
Plugging this result back into equation (1) yields
\begin{equation}\begin{split}
p(\mathbf{y}|\mathbf{X}, \mathbf{w}, \sigma^2) &
\propto \exp\left(\frac{\mathbf{y}^2 - (\bar{y}\mathbf{1})^2 -2(\mathbf{Xw}).\mathbf{y} +
(\mathbf{Xw})^2}{-2\sigma^2}\right) \\
&= \exp\left(\frac{(\mathbf{y}-\bar{y}\mathbf{1})^2 -2(\mathbf{Xw}).\mathbf{y} +
(\mathbf{Xw})^2}{-2\sigma^2}\right) \\
&= \exp\left(\frac{(\mathbf{y}-\bar{y}\mathbf{1})^2 -2(\mathbf{Xw}).(\mathbf{y}- \bar{y}\mathbf{1}) +
(\mathbf{Xw})^2}{-2\sigma^2}\right) \\
& = \exp(-\frac{1}{2\sigma^2}\|\mathbf{y} - \bar{y}\mathbf{1}_N - \mathbf{Xw}\|_2^2).
\end{split}\end{equation}
\section*{Exercises}
\setcounter{exercise}{8}
\exercise
In this exercise we use the results of section 4.3.1 to arrive at the same
formula as that of exercise 7.5:
\[
    \mathbb{E}[y|\mathbf{x}] = \bar{y} -  \mathbf{w}^T\bar{\mathbf{x}} + \mathbf{w}^T\mathbf{x}
\]
(this formula is not the exact one given in the question but they are equivalent).
First, we find the covariance matrices $\mathbf{\Sigma}_{XX}$
and $\mathbf{\Sigma}_{XY}$ of the joint distribution:
\begin{align*}
    \mathbf{\Sigma} & = \begin{pmatrix}
        \mathbf{\Sigma}_{YY} & \mathbf{\Sigma}_{YX} \\
        \mathbf{\Sigma}_{XY} & \mathbf{\Sigma}_{XX} \\
    \end{pmatrix} \\
    & = \begin{pmatrix} \mathbf{y} & \mathbf{X} \end{pmatrix}^T
    \begin{pmatrix} \mathbf{y} & \mathbf{X} \end{pmatrix} \\
    & = \begin{pmatrix}
        \mathbf{y}^T \mathbf{y} & \mathbf{y}^T \mathbf{X} \\ 
        \mathbf{X}^T \mathbf{y} & \mathbf{X}^T \mathbf{X} \\ 
    \end{pmatrix}.
\end{align*}
Thus,
\begin{equation}\begin{split}
    \mathbf{\Sigma}_{XX} = \mathbf{X}^T\mathbf{X} \\
    \mathbf{\Sigma}_{YX} = \mathbf{y}^T\mathbf{X}.
\end{split}\end{equation}
I know the question said to find $\mathbf{\Sigma}_{XY}$, but I think we are supposed to find
$\mathbf{\Sigma}_{YX}$. Finding the means $\boldsymbol{\mu}_x$ and $\boldsymbol{\mu}_y$ 
is a lot easier:
\begin{equation}\begin{split}
    \boldsymbol{\mu}_x = \bar{\mathbf{x}} = \frac{1}{N}\sum\mathbf{x}_i \\
    \mu_y = \bar{y} = \frac{1}{N}\sum y_i.
\end{split}\end{equation}

Now, we plug (3) and (4) into equation 4.69 and replace $1$ with
$y$ and $2$ with $x$.  Given $\mathbf{x}$ our prediction for $y$ is
\begin{align*}
    \mu_{y|x} & = \mu_y + \mathbf{\Sigma}_{YX} \mathbf{\Sigma}_{XX}^{-1}
    (\mathbf{x} - \bar{\mathbf{x}}) \\
    & = \bar{y} + \mathbf{y}\mathbf{X}^T(\mathbf{X}^T\mathbf{X})^{-1}
    (\mathbf{x} - \bar{\mathbf{x}}) \\
    & = \bar{y} + ((\mathbf{X}^T\mathbf{X})^{-1}\mathbf{X}^T\mathbf{y})^T
    (\mathbf{x} - \bar{\mathbf{x}}) \\
    & = \bar{y} +  - \mathbf{w}^T\bar{\mathbf{x}} + \mathbf{w}^T\mathbf{x} \\
    & = \mathbb{E}[y|\mathbf{x}].
\end{align*}
Recall that $\mathbf{X}^T\mathbf{X}$ is symmetric so $(\mathbf{X}^T\mathbf{X})^{-T} = 
(\mathbf{X}^T\mathbf{X})^{-1}$.
I find it reassuring that the discriminative and generative approach converge to the
same solution. That being said, I am not sure what the author is looking for in part b)
as in both approaches, one ends up doing the same calculations.

\end{document}

\documentclass[a4paper,11pt]{article}
\usepackage[utf8]{inputenc}
\usepackage{amssymb}
\usepackage{amsmath} 
\usepackage{enumerate}

\DeclareMathOperator*{\argmax}{arg\!\max}
\DeclareMathOperator*{\argmin}{arg\!\min}
\DeclareMathOperator*{\var}{var}
\DeclareMathOperator*{\Unif}{Unif}
\newcounter{exercise}
\setcounter{exercise}{0}
\newcounter{subexercise}
\newcommand*{\exercise}[1][]{
  \subsection*{Exercise
    \ifx/#1/\stepcounter{exercise}\arabic{exercise}
    \else#1\fi
  }
  \setcounter{subexercise}{0}
}
\newcommand*{\subexercise}[1][]{
  \par{
    \noindent\textbf{\ifx/#1/\protect\stepcounter{subexercise}\alph{subexercise}\else#1\fi.\quad}
  }
}
\title{Chapter 9}
\author{stevenjin8}
\date{\today}

\begin{document}
  \maketitle

  \section*{Comments}
  My clustering experience has been very limited to non-probabilistic methods such as k-means and
  hierarchical clustering, so I really wanted to get a good understanding of Dirichlet Processes (DP).
  However I found the notation somewhat confusing. Hopefully if my future self ever sees DP, this
  little blurb will serve as a good reminder.

  The confusion starts in section 25.2.2 where the author defines a Dirichlet Process as a
  "distribution over probability measures $G : \Theta \rightarrow \mathbb{R}^+$, where we require
  $G(\theta) \geq 0$ and $\int_\Theta G(\theta) = 0$." He then goes on to say that $(G(T_1),...,G(T_K))$
  has a join Dirichlet distribution." This made very little sense to me because $T_i \subseteq \Theta$,
  not $T_i \in \Theta$. It made even less sense in equation 25.22 in the usage of the Dirac delta,
  since it is only relevant at one point.

  What really helped me understand was learning what a measure is, and the motivation behind
  measures in probability theory. Following \cite{prob-measure}, we see that valid probability
  distributions can be quite clunky to express with pdf's and cdf's. Let $a$ be a random variable
  with a support of $\{0, 1\}$ and uniform probabilities. Now let $b$ be a random variable such
  that $b=1$ if  $a=1$, but  $b|a=0 \sim \Unif[0,1]$. The marginal cdf of $b$ has a discontinuity at 1,
  thus the pdf does not exist. In other words, despite the marginal of $b$ being a valid random
  variable, its distribution cannot be expressed in terms of a pdf (very cleanly).

  What we really want is an abstract function that gives us a probability for subsets of the support.
  More formally we want a function $G: \mathcal{A} \rightarrow [0,1]$ such that
  \begin{enumerate}
    \item $G(\Theta) = 1$,
    \item $G(S) + G(T) = G(S \cup T), S \cap T = \emptyset$.
  \end{enumerate}
  where $\mathcal{A}$ is an algebra of $\Theta$ (or a $\sigma$-algebra if $\Theta$ is continuous).
  An algebra of $\Theta$ is a set of sets that contains $\Theta$, and is closed under unions and complements.
  A $\sigma$-algebra is like an algebra, but it is also closed under countably finite unions (not
  too sure in what circumstances and algebra would not be a $\sigma$-algebra).

  Applying this to section 25.2.2, I think it would be more appropriate to say that $G$ is a
  probability measure over $\Theta$. If we let $I$ be the posterior probability measure over 
  $\Theta$ given some observations $\overline{\theta}_1, ..., \overline{\theta}_N$, with
  distinct values $\theta_1, ..., \theta_K$, then we can partition $\Theta$ into $K+1$ partitions
  $\{ \theta_1 \}, ..., \{ \theta_K \}, \Theta \setminus \{ \theta_1, ..., \theta_N \}$.
  Rewriting equation 25.27, we have
  \begin{align*}
    I(\{ \theta_k \}) &= \frac{N_k}{\alpha + N} \\
    I(\Theta \setminus \{ \theta_1, ..., \theta_K \}) &= \frac{\alpha}{\alpha + N}
  \end{align*}
  where $N_k$ is the amount of times $\theta_k$ occurs in our samples.

  The moral of the story is that, unlike pdf's, probability measures allow us to assign non-zero
  probabilities to sets of measure 0.

  My final note is that in the context of sets, $\delta_x(T) = \mathbb{I}(x \in T)$.
  \section*{Exercises}
  \exercise
  \subexercise

  \begin{thebibliography}{9}
    \bibitem{prob-measure} 
    Evans Lawrence.
    \textit{Mini Lecture \#1 - Why use measure theory for probability?}.
    \\\texttt{https://www.youtube.com/watch?v=RjPXfUT7Odo}
  \end{thebibliography}
\end{document}

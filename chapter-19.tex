\documentclass[a4paper,11pt]{article}
\usepackage[utf8]{inputenc}
\usepackage{amssymb}
\usepackage{amsmath} 
\usepackage{enumerate}

\DeclareMathOperator*{\argmax}{arg\!\max}
\DeclareMathOperator*{\argmin}{arg\!\min}
\DeclareMathOperator*{\var}{var}
\newcounter{exercise}
\setcounter{exercise}{0}
\newcounter{subexercise}
\newcommand*{\exercise}[1][]{
  \subsection*{Exercise
    \ifx/#1/\stepcounter{exercise}\arabic{exercise}
    \else#1\fi
  }
  \setcounter{subexercise}{0}
}
\newcommand*{\subexercise}[1][]{
  \par{
    \noindent\textbf{\ifx/#1/\protect\stepcounter{subexercise}\alph{subexercise}\else#1\fi.\quad}
  }
}
\title{Chapter 19}
\author{stevenjin8}
\date{\today}

\begin{document}
  \maketitle

  \section*{Comments and Proofs}
  I found this chapter a bit tough due to the sudden introduction of the node potential $\phi_t$ in
  equation 19.24. It really just means that we assign potentials both edges and nodes (unlike what
  section 19.3 suggests).
  \section*{Exercises}
  \exercise
  \begin{align*}
    \frac{\partial}{\partial\boldsymbol{\theta}_c} \log Z(\boldsymbol{\theta}) &=
    \frac{1}{Z(\boldsymbol{\theta})} \frac{\partial}{\partial\boldsymbol{\theta}_c} \left[
      \sum\limits_\mathbf{y} \exp \left(
        \boldsymbol{\theta}_c^T
        \boldsymbol{\phi}_c(\mathbf{y})
      \right)
    \right] \\
   &= \sum\limits_\mathbf{y}
  \boldsymbol{\phi}_c(\mathbf{y}) \frac{1}{Z(\boldsymbol{\theta})}
  \exp \left(
    \boldsymbol{\theta}_c^T
    \boldsymbol{\phi}_c(\mathbf{y})
  \right) \\
  &= \sum\limits_\mathbf{y} \boldsymbol{\phi}_c(\mathbf{y}) p(\mathbf{y}|\boldsymbol{\theta})
  \end{align*}

  \setcounter{exercise}{3}
  The cost of training an MRF is $O(rk(N+c)) = O(r(kN + kc))$. Looking at equation 19.41, we see that
  that we have to compute each feature for each data point per iteration. However, we only have to
  compute the marginals once per feature.

  The cost of training a CRF $O(rNk(1+c)) = O(rNkc)$. The key difference is that the unclamped term
  is now conditioned on each data point. Thus, we must calculate the marginals $N$ times (assuming
  that each data point is different). In other words, we must compute marginals for each
  iteration, data point, and feature.

  \exercise
  \begin{align*}
    p(x_i=1|\mathbf{x}_{nb_i}) &=
    \frac{p(x_i=1, \mathbf{x}_{nb_i}|\boldsymbol{\theta})}{
      p(x_i=0, \mathbf{x}_{nb_i}|\boldsymbol{\theta})
      + p(x_i=1, \mathbf{x}_{nb_i}|\boldsymbol{\theta})
    } \\
    &= \frac{e^z_i}{1 + e^z_i} \\
    & = \frac{1}{1 + e^{-z_i}}
  \end{align*}
  where
  \begin{align*}
    z_i &= p(x_i=1, \mathbf{x}_{nb_i}|\boldsymbol{\theta}) \\
    &= \frac{1}{Z(\boldsymbol{\theta})}\exp(h_i)\prod\limits_{j \in nb_i}\exp(J_{ij}x_j) \\
    &= \exp\left(\log Z(\boldsymbol{\theta}) + h_i + \sum\limits_{j\in nb_i} J_{ij}x_j\right)
  \end{align*}

  If we keep equation 19.125 intact, but switch $x_i \in \{-1, 1\}$, then we have
  \begin{equation}
    p(x_i=1|\mathbf{x}_{nb_i}, \boldsymbol{\theta}) = \frac{ z_i^+ }{z_i^- + z_i^+}
  \end{equation}
  where $z_i^+$ is the unormalized probability
  $\tilde{p}(x_i=1|\mathbf{x}_{nb_i}, \boldsymbol{\theta})$ and, similarly,
  $z_i^- = \tilde{p}(x_i=-1|\mathbf{x}_{nb_i}, \boldsymbol{\theta})$.
  Since 
  \begin{align*}
    z_i^+ &= \exp\left(h_i + \sum\limits_{j\in nb_i} J_{ij}x_j\right) \\
    z_i^- &= \exp\left(-h_i - \sum\limits_{j\in nb_i} J_{ij}x_j\right), \\
  \end{align*}
  we can see that $z_i^- = \frac{1}{z_i^-}$. Plugging back into equation 1, we get
  \begin{align*}
    p(x_i=1|\mathbf{x}_{nb_i}, \boldsymbol{\theta}) &= \frac{z_i^+}{z_i^- + z_i^+} \\
    &= \frac{1}{1 + (z_i^-)^2}.
  \end{align*}

\end{document}

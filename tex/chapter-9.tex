\documentclass[a4paper,11pt]{article}
\usepackage[utf8]{inputenc}
\usepackage{amssymb}
\usepackage{amsmath} 
\usepackage{enumerate}

\DeclareMathOperator*{\argmax}{arg\!\max}
\DeclareMathOperator*{\argmin}{arg\!\min}
\DeclareMathOperator*{\var}{var}
\newcounter{exercise}
\setcounter{exercise}{0}
\newcounter{subexercise}
\newcommand*{\exercise}[1][]{\subsection*{Exercise \ifx/#1/\stepcounter{exercise}\arabic{exercise}\else#1\fi}\setcounter{subexercise}{0}}
\newcommand*{\subexercise}[1][]{
\par{\noindent\textbf{\ifx/#1/\protect\stepcounter{subexercise}\alph{subexercise}\else#1\fi.\quad}}}

\title{Chapter 9}
\author{stevenjin8}
\date{\today}

\begin{document}
\maketitle

\section*{Comments and Proofs}
\subsection*{Section 9.2.2.1}
Note that in this context, $\mu$ is a parameter of the model,
not $\frac{1}{1+\exp(-\mathbf{w}^T\mathbf{x})}$.

\subsection*{Section 9.3.1}
Because equation 9.77 looks quite different than equation 9.2, I did not find
equations 9.81 and 9.82 trivial.

\textit{Proof of equation 9.81.} We first find $A(\theta)$.
\begin{align*}
    \int p( y|\theta_i, \sigma^2 ) &= 1 \\
    \int \exp\left[
        \frac{ y\theta_i - A(\theta_i) }{ \sigma^2 } - c(\theta_i, \sigma^2)
    \right]dy &= 1 \\
    \exp\left[ \frac{ -A(\theta_i) }{ \sigma^2 }\right]
    \int \exp\left[
        \frac{ y\theta_i }{ \sigma^2 } - c( \theta_i, \sigma^2 )
    \right]dy
    &= 1 \\
    \int \exp\left[
        \frac{ y\theta_i }{ \sigma^2 } - c( \theta_i, \sigma^2 )
    \right]dy
    &= \exp\left[ \frac{ A( \theta_i ) }{ \sigma^2 }\right] \\
    \sigma^2\log\int \exp\left[
        \frac{ y\theta_i }{ \sigma^2 } - c( \theta_i, \sigma^2 )
    \right]dy &= A( \theta ).
\end{align*}
Next, we derive in terms of $\theta$:
\begin{align*}
    \frac{ \partial A }{ \partial \theta_i }
    &= \frac{ \partial }{ \partial \theta_i }
    \sigma^2\log\int \exp\left[
        \frac{ y\theta_i }{ \sigma^2 } - c( \theta_i, \sigma^2 )
    \right]dy \\
    &= \sigma^2 \frac{
        \int \frac{ \partial }{ \partial\theta }\exp\left[
            \frac{ y\theta_i }{ \sigma^2 } - c( y_i, \sigma^2 )
        \right]dy
    }{
        \int \exp\left[
            \frac{ y\theta_i }{ \sigma^2 } - c( y_i, \sigma^2 )
        \right]dy
    } \\
    &= \sigma^2 \frac{
        \int \frac{ y }{ \sigma^2 } \exp\left[
            \frac{ y\theta_i }{ \sigma^2 } - c( y_i, \sigma^2 )
        \right]dy
    }{
        \exp( \frac{ A( \theta_i )}{ \sigma^2 })
    } \\
    &= \int y p( y|\theta_i, \sigma^2 ) dy = \mathbb{E}[ y|\theta_i, \sigma ].
\end{align*}

\textit{Proof of equation 9.8.2}.
Using the proof of equation 9.8.1, we have
\begin{align*}
    \frac{ \partial }{ \partial \theta_i }\frac{ \partial A }{ \partial \theta_i }
    &= \frac{ \partial }{ \partial \theta_i }
    \int y \exp\left[
        \frac{ y\theta_i - A( \theta_i )}{ \sigma^2 } - c( y_i, \sigma^2 )
    \right]dy \\
    &= \int y \frac{\partial}{\partial \theta_i} \exp\left[
        \frac{ y\theta_i - A(\theta_i) }{ \sigma^2 } - c( y_i, \sigma^2 )
    \right]dy \\
    &= \int y\left(
		\frac{ y-\frac{ \partial A }{ \partial \theta_i }}{ \sigma^2 }
	\right)\exp\left[
        \frac{ y\theta_i - A(\theta_i) }{ \sigma^2 } - c( y_i, \sigma^2 )
    \right]dy \\
    &= \frac1{ \sigma^2 } \int
    \left(
        y^2 - y\frac{ \partial A }{ \partial \theta_i }
    \right)p( y|\theta_i, \sigma^2 )dy \\
    &= \frac1\sigma \left(
        \int y^2p( y|\theta_i, \sigma^2 )dy
        - \frac{ \partial A }{ \partial \theta_i }
        \int y p( y|\theta_i, \sigma^2 )dy
    \right) \\
    &= \frac1\sigma^2\left(
        \mathbb{E}[ y^2|\theta_i, \sigma^2 ]
		- \mathbb{E}[ y|\theta_i, \sigma^2 ] ^ 2
    \right) \\
    &= \frac1\sigma^2 \var[ y|\theta_i, \sigma^2 ].
\end{align*}
It follows that
\[
    \var[ y|\theta_i, \sigma^2 ]
    = \sigma ^ 2 \frac{ \partial^2A }{ \partial\theta_i^2 }.
\]
\section*{Exercises}

\setcounter{exercise}{1}

\exercise
\begin{align*}
    \mathcal{N}(\mathbf{x} | \boldsymbol{\Sigma}, \boldsymbol{\theta} ) &  =
    \frac1{
		( 2\pi )^{ \fracD2 } \lvert \boldsymbol{ \Sigma } \rvert ^ { \frac12 }
    }\exp\left(
        -\frac12
        ( \mathbf{x} - \boldsymbol{\mu} )^T
        \boldsymbol{\Sigma}^{-1}
        ( \mathbf{x} - \boldsymbol{\mu} )
    \right) \\
    &= \frac1{
		( 2\pi )^{ \frac12 } \lvert \boldsymbol{\Sigma} \rvert ^ { \frac12 }
    }\exp\left(
        -\frac12 (
            \mathbf{x}^T \boldsymbol{\Sigma} ^ {-1} \mathbf{x}
            - 2\boldsymbol{\mu} ^ T \boldsymbol{\Sigma} ^ {-1} \mathbf{x}
            + \boldsymbol{\mu} ^ T \boldsymbol{\Sigma} ^ {-1} \boldsymbol{\mu}
        )
    \right) \\
    &= \frac1{
		( 2\pi ) ^ { \fracD2 } \lvert \boldsymbol{\Sigma} \rvert | ^ { \frac12 }
    }
    \exp\left(
        -\frac12 \boldsymbol{\mu} ^ T \boldsymbol{\Sigma} ^ {-1} \boldsymbol{\mu}
    \right)
    \exp\left(
        -\frac12 \mathbf{x} ^ T \boldsymbol{\Sigma} ^ {-1} \mathbf{x}
        + \boldsymbol{\mu} ^ T \boldsymbol{\Sigma} ^ {-1} \mathbf{x}
    \right) \\
    &= \frac1{A( \boldsymbol{\Sigma} , \boldsymbol{\mu} )} h( \mathbf{x} )
    \exp\left(
        \boldsymbol{\eta} ( \boldsymbol{\Sigma}, \boldsymbol{\mu} ) ^ T
        \boldsymbol{\phi} ( \mathbf{x} )
    \right)
\end{align*}
Where
\begin{align*}
    h( \mathbf{x} ) &= 1, \\
    A\left( \boldsymbol{\Sigma}, \boldsymbol{\mu} \right) &= \frac1{
		( 2\pi ) ^ { \fracD2 } \lvert \boldsymbol{\Sigma} \rvert ^ { \frac12 }
    }
    \exp\left(
        -\frac12 \boldsymbol{\mu} ^ T\boldsymbol{\Sigma} ^ {-1} \boldsymbol{\mu}
    \right),
\end{align*}
$\boldsymbol{\eta}( \boldsymbol{\Sigma}, \boldsymbol{\mu} )$ is $-\frac12$
times a concatenation of the rows (or columns) of
$\boldsymbol{\Sigma} ^ {-1}$ and $\boldsymbol{\mu} ^ T \boldsymbol{\Sigma}$,
and $\boldsymbol{\phi}( \mathbf{x} )$ is a concatenation of
the rows (or columns) of $\mathbf{x} \mathbf{x} ^ T$ and $\mathbf{x}$.
\end{document}

\documentclass[a4paper,11pt]{article}
\usepackage[utf8]{inputenc}
\usepackage{amssymb}
\usepackage{amsmath} 
\usepackage{enumerate}

\DeclareMathOperator*{\argmax}{arg\!\max}
\DeclareMathOperator*{\argmin}{arg\!\min}
\DeclareMathOperator*{\var}{var}
\newcounter{exercise}
\setcounter{exercise}{0}
\newcounter{subexercise}
\newcommand*{\exercise}[1][]{
  \subsection*{Exercise
    \ifx/#1/\stepcounter{exercise}\arabic{exercise}
    \else#1\fi
  }
  \setcounter{subexercise}{0}
}
\newcommand*{\subexercise}[1][]{
  \par{
    \noindent\textbf{\ifx/#1/\protect\stepcounter{subexercise}\alph{subexercise}\else#1\fi.\quad}
  }
}
\title{Chapter 20}
\author{stevenjin8}
\date{\today}

\begin{document}
  \maketitle

  \section*{Comments and Proofs}
  I did not retain much from this chapter...
  \section*{Exercises}
  
  \setcounter{exercise}{2}
  \exercise
  \subexercise
  We first use Bayes law to "flip" the conditionals and then compute the marginals:
  \begin{align*}
    p(G_1|X_2=50) &\propto p(X_2=50|G_1)p(G_1) \\
    &\propto p(X_2=50|G_1) \\
    &\propto p(X_2=50|G_2=1)p(G_2=1|G_1) + p(X_2=50|G_2=2)p(G_2=2|G_1)
  \end{align*}
  Plugging in the numbers, we get $p(G_1|X_2=50) = [0.895,1.05]$.

  \subexercise
  The key realization here is that $p(X_2|G1)=p(X_3|G_1)$ since the transition matrices are equal.
  Thus,
  \begin{align*}
    p(G_1=1|X_2=50, X_3=50) &= p(X_2=50, X_3=50|G_1=1) \\
    &= p(X_2=50|G_1=1)p(X_3=50|G_1=1) \\
    &= p(X_2=50|G_1=1) ^ 2, \\
    p(G_1|X_2=50, X_3=50) &= [0.986, 0.014].
  \end{align*}
  It makes sense that $p(G_1=1|X_2=50,X_3=50) > p(G_1=1|X_2=50)$, since in the former case,
  there is more evidence for $G_1=1$.
  
  \subexercise
  Now, there is equal evidence that $G=0$ as there was for $G=1$ in part b, and the prior
  distribution for $G_1$ is uniform. Thus, we can flip the distribution and
  $p(G_1|X_2=60, X_3=60) = [0.014, 0.986]$.

  \subexercise
  I think the author meant to ask for $p(G_1|X_2=50,X_3=60)$. If that is the case, there
  is equal evidence for $G=1$ and $G=0$. Thus, the posterior of $G_1$ is just its prior:
  $p(G_1|X_2=50,X_3=60) = [0.5, 0.5]$.
\end{document}
